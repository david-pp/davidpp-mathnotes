\documentclass[11pt,a4paper]{report}
\usepackage{indentfirst}   %% 首行缩进
\usepackage{fontspec, xunicode, xltxtra, verbatim}
\usepackage{listings}      %% 支持代码
\usepackage{xcolor}        %% 代码颜色
\usepackage{amssymb}
\usepackage{amsmath}
\usepackage{makeidx}
\usepackage{tikz}
\usepackage{tikz-3dplot}
\usetikzlibrary{calc,patterns,angles,quotes}

\usepackage{fancyhdr} %% 定制页眉页脚
\pagestyle{fancy}

%\setmainfont{Microsoft YaHei}
\setmainfont{Heiti SC}  
%\setmainfont{SimSun}       %% 字体

\XeTeXlinebreaklocale zh   %% 中文自动换行
\XeTeXlinebreakskip = 0pt plus 1pt
\parindent 2em             %% 段落缩进
\linespread{1.2}           %% 行间距

\makeindex %% 生成索引

%% 控制verbatim里面显示中文
\makeatletter
\def\verbatim@font{\rmfamily\small} %如果使用roman字体族,将sffamily改成rmfamily
\makeatother

\author{David}
\title{游戏中的数学}

%% 超链接高亮
\usepackage{hyperref}
\hypersetup{colorlinks=false}

\begin{document} 
\maketitle
\tableofcontents


\chapter{Vector}
\section{Defination}

一般Vector翻译为:

\begin{itemize}
\item 向量  - 代数
\item 矢量  - 几何
\end{itemize}

 \subsection{代数定义}

行向量(Row Vector):
\[
\mathbf{v} = \vec{v} =   \begin{bmatrix}
v_x & v_y  &  v_z
\end{bmatrix}
\]

列向量(Column Vector):
\[
\mathbf{v} = \vec{v} =   \begin{bmatrix}
v_x \\
v_y \\
v_z
\end{bmatrix}
\]

\subsection{几何定义}

\begin{center}
\begin{tikzpicture}
%\filldraw[black] (0,0) circle (2pt) node[anchor=west] {Intersection point};
%\draw[gray, thick] (-1,2) -- (2,-4);
%\draw[gray, thick] (-1,-1) -- (2,2)
\filldraw[black] (0,0) circle (1pt) node[anchor=west] {Tail};
\filldraw[black] (4,4) circle (0pt) node[anchor=west] {Head};
\draw[gray, thick,->] (0,0) -- (4,4);
\end{tikzpicture}
\end{center}


\begin{itemize}
\item 大小(Magnitude):

\[
|\vec{v}| = \sqrt{v_x^2 + v_y^2 + v_z^2} = \sqrt{\vec{v}\cdot\vec{v}}
\]

\item 方向(Direction):可以用单位向量表示(Normalized)

\[
\hat{v} =\frac{\vec{v}}{|\vec{v}|}
\]


\end{itemize}


\subsection{用途}

\begin{itemize}
\item 位移(Displacement)
\item 速度(Velocity)
\item \ldots
\end{itemize}


\section{Vector vs.  Point}

意义的区别:

\begin{itemize}
\item Point - 空间绝对位置
\item Vector - 空间相对位置
\end{itemize}


等价关系:

\begin{center}
\begin{tikzpicture}
    % Draw axes
    \draw [<->,thick] (0,4) node (yaxis) [above] {$y$}
        |- (4,0) node (xaxis) [right] {$x$};
    % Draw Point
    \filldraw[red, right] (3,2) circle (1pt) node[anchor=west] {$\mathbf{p(x,y)}$};
    % Draw Vector
    \draw[gray,  very thick,->] (0,0) -- (3,2);
    \draw[thin, ->,  dashed] (3,1) -- (1.6,1);
    \node[right] at (3,1) {$\vec{v} = (x,y)$};
\end{tikzpicture}
\end{center}


尽管代数上是等价的,几何要进行区分理解:

\begin{itemize}
\item  $\vec{v}$,给出列从原点到点$p(x,y,z)$的位移
\item  $p(x,y,z)$,则是从原点开始沿着$\vec{v}$指定的量进行移动,最终到达的位置
\end{itemize}


\section{运算及其几何意义}

分两个角度来看:一个代数(计算)角度,一个几何(图形)解释。

向量运算有:
\begin{itemize}
\item 加法 : 几何运算满足三角规则(Triangle Rule)
\begin{itemize}
\item 减法:减法是加法的逆运算
\item 逆元:$\vec{v}$ 的逆元是 $-\vec{v}$,几何意识是反方向
\end{itemize}

\item 数乘:缩放向量$\vec{v}$
\item 点积:可用两种几何视角来看,当作投影或向量夹角
\item 叉积:计算相交两条线的法线

\end{itemize}

\subsection{+}


\subsubsection{代数定义}

\[
\vec{a} + \vec{b}  = 
\begin{bmatrix}
a_x\\
a_y\\
a_z 
\end{bmatrix} + 
\begin{bmatrix}
b_x\\
b_y\\
b_z 
\end{bmatrix}
= 
\begin{bmatrix}
a_x + b_x\\
a_y + b_y\\
a_z + b_z 
\end{bmatrix}
\]

满足规则:

\begin{itemize}
\item $\vec{a} + \vec{b} = \vec{b} + \vec{a}$
\item $\vec{a} - \vec{b} = \vec{a} + (-\vec{b}) = - (\vec{b} - \vec{a})$
\end{itemize}


\subsubsection{几何解释}

满足三角形法则(Triangle Rule):

\begin{center}
\begin{tikzpicture}
\draw [help lines] (0,-1) grid (15,6);
%
% a = (4, 1) 
% b = (-2, 3)
% a+b = (2,4)
% 
% Triangle Rule for Addition
\coordinate (A) at (0,0) {};
\coordinate (B) at (4,1) {};
\coordinate (C) at (2,4) {}; 
\coordinate (D) at (6,5) {};
\draw [thick,  ->,  arrows={-latex}] (A) -- (B);
\draw [thick,  ->, arrows={-latex}] (B) -- (C);
\draw [thick,  ->, arrows={-latex}] (A) -- (C);
\draw [thick,  ->, arrows={-latex},  dashed] (C) -- (D);
\draw [thick,  ->, arrows={-latex},  dashed] (B) -- (D);
\node [below] at(2,0.5) {$\vec{a}$};
\node [above] at(4,4.5) {$\vec{a}$};
\node [right] at(3,2.5) {$\vec{b}$};
\node [left] at(1,2) {$\vec{a}+\vec{b}$};
\node [right] at(5,3) {$\vec{b}+\vec{a}$};

% Triangle Rule for Substraction
\coordinate (C) at (10,0) {};
\coordinate (D) at (14,1) {};
\coordinate (E) at (8,3) {}; 

\draw [thick,  ->,  arrows={-latex}] (C) -- (D);  % a
\draw [thick,  ->,  arrows={-latex}] (C) -- (E);  % a
\draw [thick,  ->,  arrows={-latex}] (E) -- (D);  % a-b
\draw [thick,  ->,  arrows={-latex}, dashed, red] (14, 1.3) -- (8,3.3);  % b-a
\node [below] at(12,0.5) {$\vec{a}$};
\node [left] at(9,1.5) {$\vec{b}$};
\node [below] at(11,2) {$\vec{a}-\vec{b}$};
\node [above,  red] at(11,2.3) {$\vec{b}-\vec{a}$};

\end{tikzpicture}
\end{center}


\subsubsection{常见用途}

\begin{itemize}
\item 加法:多次位移的计算
\item 减法:从一个点到另一个点的位移矢量
\end{itemize}

\subsection{k}
\subsubsection{代数定义}

\[
k\vec{v} = k
\begin{bmatrix}
v_x \\
v_y \\
v_z
\end{bmatrix}
 = 
\begin{bmatrix}
kv_x \\
kv_y \\
kv_z
\end{bmatrix}
\]

满足规则:

\begin{itemize}
\item $k(\vec{a} + \vec{b}) = k\vec{a} + k\vec{b}$
\end{itemize}


\subsubsection{几何解释}

可以用于向量的缩放:

\begin{center}
\begin{tikzpicture}
\draw [help lines] (0,0) grid (14,9);
%v
\draw [thick, ->] (0,0) node [below] {$\vec{v}$} -- (2,4);
%2v
\draw [thick, ->] (2,0)  node [below] {$2\vec{v}$} -- (6,8);
%v/2
\draw [thick, ->] (4,0)  node [below] {$\vec{v}/2$}  -- (5,2);
%-v
\draw [thick, <-] (6,0) node [below] {$\vec{v}$} -- (8,4);
%-v/2
\draw [thick, <-] (8,0)  node [below] {$2\vec{v}$} -- (12,8);
%-2v
\draw [thick, <-] (10,0)  node [below] {$2\vec{v}$} -- (14,8);
\end{tikzpicture}
\end{center}

\subsection{dot}

\subsubsection{代数定义}

\[
\vec{a}\cdot\vec{b} = 
\begin{bmatrix}
a_x \\
a_y \\
a_z
\end{bmatrix}
\cdot
\begin{bmatrix}
b_x \\
b_y \\
b_z
\end{bmatrix} =
a_xb_x + a_yb_y + a_zb_z
\]

满足规则:

\begin{itemize}
\item 点积交换律:$\vec{a}\cdot\vec{b} = \vec{b}\cdot\vec{a}$
\item 点积数乘分配律:$(k\vec{b})\cdot\vec{a} = k(\vec{b}\cdot\vec{a})  = \vec{b}\cdot(k\vec{a})$ 
\item 点积加法分配律: $(\vec{b}+\vec{c})\cdot\vec{a} = \vec{b}\cdot\hat{a} + \vec{c}\cdot\vec{a}$ 
\end{itemize}

\subsubsection{几何解释 - 投影}

投影视角解释(Projection):

点积$\vec{a}\cdot\vec{b}$等于$\vec{b}$投影到平行于$\vec{a}$的任何一条线上的有符号长度(Singed Distance),乘以$\vec{a}$的长度。


\begin{center}
\begin{tikzpicture}
\draw [help lines] (0,-1) grid (9,5);

% a=(1,0)
% b=(4,2)
\coordinate (O) at (0,0) {};
\coordinate (A) at (1, 0) {};
\coordinate (B_Tail) at (2, 2) {};
\coordinate (B_Head) at (6, 4) {};
\coordinate (Dot_Tail) at (2, 0) {};
\coordinate (Dot_Head) at (6, 0) {};

\draw [thick,  ->,  arrows={-latex}] (O) -- (A);
\draw [thick,  ->,  arrows={-latex}] (B_Tail) -- (B_Head);
\draw [thick,  ->,  arrows={-latex},  red] (Dot_Tail) -- (Dot_Head);
\draw [red, dashed, thick] (B_Tail) -- (Dot_Tail);
\draw [red, dashed, thick] (B_Head) -- (Dot_Head);
\node [below] at (0.5, 0) {$\hat{a}$};
\node [above] at (4, 3.5) {$\vec{b}$};
\node [below] at (4, 0) {$\hat{a}\cdot\vec{b} = \vec{b}\cdot\hat{a}$ };
\draw [thick] (2,-.2) -- (2,0.1);
\draw [thick] (6,-.2) -- (6,0.1);

\end{tikzpicture}
\end{center}

有符号投影长度:

\begin{center}
\begin{tikzpicture}[scale=0.7]
\draw [help lines] (0,-1) grid (20,5);

% a=(1,0)
% b=(4,2)
% c=(0,2)
% d=(-3, 3)
\coordinate (O) at (0,0) {};
\coordinate (A) at (1, 0) {};
\coordinate (B_Tail) at (2, 2) {};
\coordinate (B_Head) at (6, 4) {};
\coordinate (Dot_Tail) at (2, 0) {};
\coordinate (Dot_Head) at (6, 0) {};

% a
\draw [thick,  ->,  arrows={-latex}] (O) -- (A);

% a.b > 0
\draw [thick,  ->,  arrows={-latex}] (B_Tail) -- (B_Head);
\draw [thick,  ->,  arrows={-latex},  red] (Dot_Tail) -- (Dot_Head);
\draw [red, dashed, thick] (B_Tail) -- (Dot_Tail);
\draw [red, dashed, thick] (B_Head) -- (Dot_Head);
\node [below] at (0.5, 0) {$\hat{a}$};
\node [above] at (4, 3.5) {$\vec{b}$};
\node [below] at (4, 0) {$\vec{b}\cdot\hat{a} > 0$ };
\draw [thick] (2,-.2) -- (2,0.1);
\draw [thick] (6,-.2) -- (6,0.1);

% a.c = 0
\coordinate (C_Tail) at (10, 2) {};
\coordinate (C_Head) at (10,4) {};
\draw [thick,  ->,  arrows={-latex}] (C_Tail) -- (C_Head);
\draw [red, dashed, thick] (C_Tail) -- (10,0);
\filldraw [] (10,0) circle (2pt) node [below] {$\vec{c}\cdot\hat{a} = 0$};
\node [right] at (10, 3) {$\vec{c}$};

% a.d < 0
\coordinate (D_Tail) at (16, 2) {};
\coordinate (D_Head) at (13,5) {};
\coordinate (Dot2_Tail) at (16, 0) {};
\coordinate (Dot2_Head) at (13,0) {};
\draw [thick,  ->,  arrows={-latex}] (D_Tail) -- (D_Head);
\draw [thick,  ->,  arrows={-latex},  red] (Dot2_Tail) -- (Dot2_Head);
\draw [red, dashed, thick] (D_Tail) -- (Dot2_Tail);
\draw [red, dashed, thick] (D_Head) -- (Dot2_Head);
\node [above] at (14.5, 3.5) {$\vec{d}$};
\node [below] at (14.5, 0) {$\vec{d}\cdot\hat{a} < 0$ };
\draw [thick] (13,-.2) -- (13,0.1);
\draw [thick] (16,-.2) -- (16,0.1);


\end{tikzpicture}
\end{center}

缩放(数乘分配律):

\begin{center}
\begin{tikzpicture}[scale=0.7]
\draw [help lines] (0,-1) grid (20,5);

% a=(1,0)
% b=(4,2)
% d=(8,4) = 2b
\coordinate (O) at (0,0) {};
\coordinate (A) at (1, 0) {};
\coordinate (B_Tail) at (2, 2) {};
\coordinate (B_Head) at (6, 4) {};
\coordinate (Dot_Tail) at (2, 0) {};
\coordinate (Dot_Head) at (6, 0) {};

% a
\draw [thick,  ->,  arrows={-latex}] (O) -- (A);

% a.b > 0
\draw [thick,  ->,  arrows={-latex}] (B_Tail) -- (B_Head);
\draw [thick,  ->,  arrows={-latex},  red] (Dot_Tail) -- (Dot_Head);
\draw [red, dashed, thick] (B_Tail) -- (Dot_Tail);
\draw [red, dashed, thick] (B_Head) -- (Dot_Head);
\node [below] at (0.5, 0) {$\hat{a}$};
\node [above] at (4, 3.5) {$\vec{b}$};
\node [below] at (4, 0) {$\vec{b}\cdot\hat{a}$ };
\draw [thick] (2,-.2) -- (2,0.1);
\draw [thick] (6,-.2) -- (6,0.1);

% a.kb
\coordinate (D_Tail) at (8, 0) {};
\coordinate (D_Head) at (16,4) {};
\coordinate (Dot2_Tail) at (8, 0) {};
\coordinate (Dot2_Head) at (16,0) {};
\draw [thick,  ->,  arrows={-latex}] (D_Tail) -- (D_Head);
\draw [thick,  ->,  arrows={-latex},  red] (Dot2_Tail) -- (Dot2_Head);
\draw [red, dashed, thick] (D_Tail) -- (Dot2_Tail);
\draw [red, dashed, thick] (D_Head) -- (Dot2_Head);
\node [above] at (12.5, 2.5) {$k\vec{b}$};
\node [below] at (12.5, 0) {$(k\vec{b})\cdot\hat{a} = k(\vec{b}\cdot\hat{a})  = \vec{b}\cdot(k\hat{a})$ };
\draw [thick] (8,-.2) -- (8,0.1);
\draw [thick] (16,-.2) -- (16,0.1);

\end{tikzpicture}
\end{center}

分配和筛选(加法分配律):

\begin{center}
\begin{tikzpicture}[scale=0.7]
\draw [help lines] (0,-2) grid (10,5);

% a=(1,0)
% b=(4,1)
% c=(2,3)
\coordinate (O) at (0,0) {};
\coordinate (A) at (1, 0) {};
\coordinate (B_Tail) at (2, 0) {};
\coordinate (B_Head) at (6, 1) {};
\coordinate (C_Tail) at (6,1) {};
\coordinate (C_Head) at (8, 4) {};
\coordinate (DotAB_Tail) at (2, 0) {};
\coordinate (DotAB_Head) at (6, 0) {};
\coordinate (DotAC_Tail) at (6, 0) {};
\coordinate (DotAC_Head) at (8, 0) {};

% a
\draw [thick,  ->,  arrows={-latex}] (O) -- (A);

% a.b
\draw [thick,  ->,  arrows={-latex}] (B_Tail) -- (B_Head);
\draw [thick,  ->,  arrows={-latex},  red] (DotAB_Tail) -- (DotAB_Head);
\draw [red, dashed, thick] (B_Tail) -- (DotAB_Tail);
\draw [red, dashed, thick] (B_Head) -- (DotAB_Head);
\node [below] at (0.5, 0) {$\hat{a}$};
\node [above] at (4, 0.5) {$\vec{b}$};
\node [below] at (4, 0) {$\vec{b}\cdot\hat{a}$ };
\draw [thick] (2,-.5) -- (2,0.1);
\draw [thick] (6,-.5) -- (6,0.1);

% a.c
\draw [thick,  ->,  arrows={-latex}] (C_Tail) -- (C_Head);
\draw [thick,  ->,  arrows={-latex},  red] (DotAC_Tail) -- (DotAC_Head);
\draw [red, dashed, thick] (C_Tail) -- (DotAC_Tail);
\draw [red, dashed, thick] (C_Head) -- (DotAC_Head);
\node [right] at (7, 2.5) {$\vec{c}$};
\node [above] at (5, 2) {$\vec{b} + \vec{c}$ };
\node [below] at (7, 0) {$\vec{c}\cdot\hat{a}$ };
\draw [thick] (8,-.5) -- (8,0.1);


% a.(b+c)
\draw [thick,  ->,  arrows={-latex}] (B_Tail) -- (C_Head);
\draw [thick,  ->,  arrows={-latex},  blue] (2,-1) -- (8,-1);
\node [below] at (5, -1) {$(\vec{b}+\vec{c})\cdot\hat{a} = \vec{b}\cdot\hat{a} + \vec{c}\cdot\hat{a}$ };


\end{tikzpicture}
\end{center}

矢量分解:

\begin{center}
\begin{tikzpicture}
\draw [help lines] (0,-1) grid (11,3);

% a=(1,0)
% b=(4,2)
\coordinate (O) at (0,0) {};
\coordinate (A) at (1, 0) {};
\coordinate (B_Tail) at (2, 0) {};
\coordinate (B_Head) at (6, 2) {};
\coordinate (Dot_Tail) at (2, 0) {};
\coordinate (Dot_Head) at (6, 0) {};

\draw [thick,  ->,  arrows={-latex}] (O) -- (A);
\draw [thick,  ->,  arrows={-latex}] (B_Tail) -- (B_Head);
\draw [thick,  ->,  arrows={-latex},  red] (Dot_Tail) -- (Dot_Head);
\draw [thick,  ->,  arrows={-latex},  blue] (Dot_Head) -- (B_Head);
\node [below] at (0.5, 0) {$\hat{a}$};
\node [above] at (4, 1) {$\vec{b}$};
\node [below] at (4, 0) {$\vec{b}_\parallel  = (\vec{b}\cdot\hat{a})\hat{a}$ };
\node [right] at (6, 1) {$\vec{b}_\perp = \vec{b} - \vec{b}_\parallel = \vec{b} - (\vec{b}\cdot\hat{a})\hat{a}$ };


\end{tikzpicture}
\end{center}

\subsubsection{几何解释 - 夹角}

\[
\cos\theta = \frac{adjacent}{hypotenuse} = \frac{\vec{b}\cdot\vec{a}}{|\vec{b}||\vec{a}|} = \frac{\vec{b}\cdot\hat{a}}{|\vec{b}|} = \hat{a}\cdot\hat{b}
\]


\[
\theta = \arccos(\frac{\vec{b}\cdot\vec{a}}{|\vec{b}||\vec{a}|})  = \arccos(\hat{a}\cdot\hat{b})
\]

\begin{center}
\begin{tikzpicture}
\draw [help lines] (0,-1) grid (11,3);

% a=(1,0)
% b=(4,2)
\coordinate (O) at (0,0) {};
\coordinate (A) at (1, 0) {};
\coordinate (B_Tail) at (2, 0) {};
\coordinate (B_Head) at (6, 2) {};
\coordinate (Dot_Tail) at (2, 0) {};
\coordinate (Dot_Head) at (6, 0) {};

\draw [thick,  ->,  arrows={-latex}] (O) -- (A);
\draw [thick,  ->,  arrows={-latex}] (B_Tail) -- (B_Head);
\draw [thick,  ->,  arrows={-latex},  red] (Dot_Tail) -- (Dot_Head);
\draw [thick,  dashed,  red] (Dot_Head) -- (B_Head);
\node [below] at (0.5, 0) {$\hat{a}$};
\node [above] at (4, 1) {$\vec{b}$};
\node [below] at (4, 0) {$\hat{a}\cdot\vec{b} = \vec{b}\cdot\hat{a}$ };
\draw [thick] (2,-.2) -- (2,0.1);
\draw [thick] (6,-.2) -- (6,0.1);
% angle
\pic [draw, ->, "$\theta$", angle eccentricity=1.5] {angle = Dot_Head--Dot_Tail--B_Head};

\end{tikzpicture}
\end{center}


不考虑$\theta$值时,可以通过点积知道两个向量之间的夹角关系:

\begin{center}
\begin{tabular}{ | c | c | c | l | } 
\hline
$\vec{a}\cdot\vec{b}$ & $\theta$ &\textbf{ 角度} & \textbf{描述} \\
\hline
$>0$ & $[0,  \pi/2)$ &  锐角 & 主要指向通一方向 \\
\hline
$=0$ & $\pi/2$ & 直角 & 垂直 \\
\hline
$<0$ & $(\pi/2,  \pi]$&  钝角 & 主要指向相反方向\\
\hline
\end{tabular}
\end{center}



\subsubsection{常见用途}

\begin{itemize}
\item 投影计算、坐标筛选计算($\vec{b}\cdot\hat{x}$)
\item 垂直判定:$\vec{a}\cdot\vec{b} = 0$

\item 半空间判定:$\vec{x}$为空间任意一点,$\vec{n}$平面法向量
\begin{itemize}
\item 平面上半空间:$\vec{x}\cdot\vec{n} > 0$
\item 平面内:$\vec{x}\cdot\vec{n} = 0$
\item 平面下半空间:$\vec{x}\cdot\vec{n} < 0$
\end{itemize}
 
\item 计算向量大小:$|\vec{a}| =\sqrt{ \vec{a}\cdot\vec{a}}$
\item 矢量分解:$ \vec{b}  =  \vec{b}_\parallel  + \vec{b}_\perp$
\item 夹角计算:$\theta = \arccos(\frac{\vec{b}\cdot\vec{a}}{|\vec{b}||\vec{a}|})  = \arccos(\hat{a}\cdot\hat{b})$
\end{itemize}

\subsection{cross}

\subsubsection{代数定义}

\[
\vec{a} \times \vec{b} = 
\begin{bmatrix}
a_x \\
a_y \\
a_y 
\end{bmatrix} \times
\begin{bmatrix}
b_x \\
b_y \\
b_y 
\end{bmatrix} = 
\begin{vmatrix}
\vec{i} & \vec{j} & \vec{k} \\
a_x & a_y & a_z \\
b_x & b_y & b_z
\end{vmatrix} = 
\begin{bmatrix}
a_yb_z - a_zb_y \\
a_zb_x - a_xb_z \\
a_xb_y - a_yb_x 
\end{bmatrix}
\]

满足规律:

\begin{itemize}
\item 叉积反交换律: $\vec{a}\times\vec{b} = - \vec{b}\times\vec{a}$
\item $\vec{a} \times \vec{a} = 0$
\item $k(\vec{a} \times \vec{b}) = (k\vec{a}) \times \vec{b} = \vec{a} \times (k\vec{b})$
\item $\vec{a} \times (\vec{b} + \vec{c}) = \vec{a} \times \vec{b} + \vec{a} \times \vec{c}$
\end{itemize}

\subsubsection{几何解释}

叉积向量的大小:包含$\vec{a}$和$\vec{b}$的平行四边形面积,为0则两条线平行

\[
|\vec{a} \times \vec{b}| = |\vec{a}| |\vec{b}| \sin\theta
\]

叉积向量的方向:

A、右手法则情况下

\begin{center}
\begin{tikzpicture}
\draw[-,fill=white!95!red](0,0)--(3,0)--(4,1)--(1,1)--cycle;
\node at (2,0.5) {$|\textcolor{blue}{\vec{a}}\times \textcolor{red}{\vec{b}}|$};
\draw[ultra thick,-latex,blue](0,0)--(3,0)node[midway,below]{$\vec{a}$};
\draw[ultra thick,-latex,red](0,0)--(1,1)node[midway,above]{$\vec{b}$};
\draw[ultra thick,-latex,blue!50!red](0,0)--(0,3)node[pos=0.7,right]{$\vec{a}\times \vec{b}$};
\draw[ultra thick,-latex,blue!50!red](0,0)--(0,-3)node[pos=0.7,right]{$\vec{b}\times \vec{a}$};
\draw (0.6,0) arc [start angle=0,end angle=45,radius=0.6]
node[pos=0.7,right]{$\theta$};
\end{tikzpicture}
\end{center}


B、左手法则情况下
\begin{center}
\begin{tikzpicture}
\draw[-,fill=white!95!red](0,0)--(3,0)--(4,1)--(1,1)--cycle;
\node at (2,0.5) {$|\textcolor{blue}{\vec{a}}\times \textcolor{red}{\vec{b}}|$};
\draw[ultra thick,-latex,blue](0,0)--(3,0)node[midway,below]{$\vec{a}$};
\draw[ultra thick,-latex,red](0,0)--(1,1)node[midway,above]{$\vec{b}$};
\draw[ultra thick,-latex,blue!50!red](0,0)--(0,3)node[pos=0.7,right]{$\vec{b}\times \vec{a}$};
\draw[ultra thick,-latex,blue!50!red](0,0)--(0,-3)node[pos=0.7,right]{$\vec{a}\times \vec{b}$};
\draw (0.6,0) arc [start angle=0,end angle=45,radius=0.6]
node[pos=0.7,right]{$\theta$};
\end{tikzpicture}
\end{center}


坐标基向量的叉积(左右手系不影响计算结果):

\begin{center}
\begin{tabular}{cc}
$\hat{x} \times \hat{y} = \hat{z}$ & $ \hat{y} \times \hat{x} = -\hat{z}$ \\
$\hat{y} \times \hat{z} = \hat{x}$ & $ \hat{z} \times \hat{y} = -\hat{x}$ \\
$\hat{z} \times \hat{x} = \hat{y}$ & $ \hat{x} \times \hat{z} = -\hat{y}$ 
\end{tabular}
\end{center}


\section{note}

\chapter{Matrix}
\section{Rotation}
\section{Scale}
\section{Transform}

\chapter{Rotation}
\section{Axis angle}
\section{Euler angle}
\section{Quaterrnion}
\section{Convertion}



\end{document}  
